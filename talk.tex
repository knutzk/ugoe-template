\documentclass[10pt]{beamer}

% In general, you should encode all files using UTF-8
\usepackage[utf8]{inputenc}
\usepackage[T1]{fontenc}

% Include SIunits to get \unit for unit typesetting
\usepackage[squaren]{SIunits}
% Set the math mode font to sans-serif
\let\mathrm\mathsf

% Use this for better font scaling (esp. if you want to use \tiny)
\usepackage{lmodern}
\usepackage{exscale}

% You can safely remove this package, it is just for displaying a layout
% page
\usepackage{layout}

% Enable rounded corners for standard blocks:
% \setbeamertemplate{blocks}[rounded]

% Add some paragraph spacing:
%\setlength{\parskip}{\smallskipamount} 

\title{Example Template}
\date{2014-06-18}
\author[Johannes Agricola]{Johannes Agricola}

\usetheme{ugoe}

\def\insertlogos{%
  \includegraphics[height=1.3cm]{logos/BMBF-Gef-Logo.pdf}%
  \hspace{1cm}%
  \includegraphics[height=1.3cm]{logos/FSP101-Atlas-Logo.pdf}%
  \hspace{1cm}%
  \includegraphics[height=1.3cm]{logos/helmholtz_logo.pdf}%
  \hspace{1cm}%
  \includegraphics[height=1.3cm]{logos/logo3.pdf}%
}

\begin{document}

\begin{frame}
  \titlepage
\end{frame}

\begin{frame}
  \frametitle{Test Slide}

  \begin{itemize}
    \item 
      Some example text
      \begin{itemize}
        \item 
          And some subitems
      \end{itemize}
  \end{itemize}

  \begin{alertblock}{Titlepage peculiarity}
    Do not forget to build this template twice after you have altered
    the title page. It uses tikz for alignment and the 
    \texttt{remember picture} option which requires two runs.

    If you forget to do that, parts of the title page may be shifted.
  \end{alertblock}
\end{frame}

\begin{frame}
  \frametitle{Test slide with a slightly longer frame title}

  \begin{itemize}
    \item 
      Long frame titles should not be used. 
    \item
      Nevertheless, they are possible now.
  \end{itemize}
\end{frame}
\begin{frame}
  \frametitle{Shorter Title}
  \framesubtitle{And a slightly longer subtitle.}

  \begin{itemize}
    \item 
      Long frame titles should not be used. 
    \item
      Nevertheless, they are possible now.
  \end{itemize}
\end{frame}

\begin{frame}{Some Blocks\ldots}
  \begin{block}{Standard Blocks}
    \textbackslash{}begin\{block\}\ldots
  \end{block}
  \begin{alertblock}{Alerted Blocks}
    \textbackslash{}begin\{alertblock\}\ldots
  \end{alertblock}
  \begin{exampleblock}{Example Blocks}
    \textbackslash{}begin\{exampleblock\}\ldots
  \end{exampleblock}
\end{frame}

\begin{frame}
  \begin{itemize}
    \item 
      A completely titleless page
  \end{itemize}
\end{frame}

\begin{frame}
  \frametitle{Did you know\ldots}
  that the ATLAS Pixel Detector extends to about $\unit{1.3}{\metre}$ in $Z$ 
  direction?  Most pixels are $\unit{50\times 400}{\micro\metre\squared}$ in 
  size.
  The area of each pixel is then the product of these sides $A = x \cdot y$.

  Sorry if you did not learn that much. This slide is mostly just for testing 
  math fonts. (By the way\ldots don't forget to use SIunits wherever possible.
  Non-italicized $\mu$s are also available when using
  \begin{center}
    \textbackslash{}unit\{400\}\{\textbackslash{}micro\textbackslash{}metre\}
    $\mapsto \unit{400}{\micro\metre}$.
  \end{center}
  Additionally, \textbackslash{}micro is available in plain text (\micro), without
  the need to include textcomp.) 

  We also have some aligned math:
  \begin{align*}
    \int_{V} {\rm div}\vec F ~{\rm d}V = \oint_{S} \vec F \cdot \vec n ~\text dS
  \end{align*}
\end{frame}

\begin{frame}
  \frametitle{Multiple columns?}
  \framesubtitle{Two Columns}
  \begin{columns}
    \begin{column}{0.5\textwidth}
      \begin{itemize}
        \item 
          This is the right column to put stuff into
        \item
          They are both centered, as usual.
        \item
          You can also put images in one of the two columns if you prefer
          that.
      \end{itemize}
    \end{column}
    \begin{column}{0.5\textwidth}
      \begin{itemize}
        \item 
          This is the right column
      \end{itemize}
    \end{column}
  \end{columns}
\end{frame}

\begin{frame}
  \frametitle{Multiple columns?}
  \framesubtitle{Two columns and some stuff below}
  \begin{columns}
    \begin{column}{0.5\textwidth}
      \begin{itemize}
        \item 
          This is the right column to put stuff into
        \item
          They are both centered, as usual.
        \item
          You can also put images in one of the two columns if you prefer
          that.
      \end{itemize}
    \end{column}
    \begin{column}{0.5\textwidth}
      \begin{itemize}
        \item 
          This is the right column
      \end{itemize}
    \end{column}
  \end{columns}
  \begin{alertblock}{}
    We can also add stuff after the two columns. Though, this does not look 
    as nice as you would expect.
  \end{alertblock}
\end{frame}

\begin{frame}
  \frametitle{Fonts}
  \tiny tiny\\
  \scriptsize scriptsize\\
  \footnotesize footnotesize\\
  \small small\\
  \normalsize normalsize\\
  \large large\\
  \Large Large \\
  \huge huge \\
  \Huge Huge\\
\end{frame}

\begin{frame}
  \layout
\end{frame}

\begin{frame}
  \frametitle{\LaTeX{} Makefile}
  This template comes with convenient latex makefile. Some Features:
  \begin{itemize}
    \item 
      Automatic SVG $\rightarrow$ PDF conversion
      \begin{itemize}
        \item 
          For files in images/*.svg
        \item
          Requires inkscape
        \item
          Can be easily extended to other file types
      \end{itemize}
    \item
      Automatic bibtexing
    \item
      Cleaning mechanism
    \item
      Keeps temporary files in tmp/
  \end{itemize}
  \begin{exampleblock}{Usage}
    Just copy the Makefile to your \LaTeX{} working directory and change
    the value of the SRC variable at its top to your top tex file. Then
    just type make in a shell in that directory.
  \end{exampleblock}
\end{frame}

\begin{frame}
  \frametitle{\LaTeX{} Makefile}
  \framesubtitle{vim Addendum}
  If you are using vim and Makefiles you may consider adding this to your 
  vimrc:
  \begin{itemize}
    \item 
      map <C-M> :!make<CR>
    \item
      incidentally, C-M is also mapped to the return key. So every time
      you will hit return and are not in edit mode, make will be called.
  \end{itemize}
\end{frame}

\appendix
\begin{frame}
  \frametitle{~}
  \begin{center}
    \textbf{\Large Thank you for your attention.}
  \end{center}
\end{frame}

\begin{frame}
  \begin{center}
    \Huge Backup
  \end{center}
\end{frame}

\begin{frame}
  \begin{alertblock}{Frame numbering}
    Have you noticed that the total page count is not counting past the thank 
    you page? This is automatically included in the ugoe theme!
  \end{alertblock}
\end{frame}

\end{document}
